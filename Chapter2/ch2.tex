\chapter{Literature Survey}
\label{ch:LR}

\begin{itemize}
\item Fungicide sprays, disease-specific chemical treatments, and vector management, for
example, could give early indications of crop health and pesticide disease
identification. This could aid in illness prevention and increased production. In this
paper, the writers investigate, analyse, and acknowledge the need for a rapid, cost-
effective, and reliable agricultural-progress monitoring sensor. They looked at current
technologies like as spectroscopic and imaging methods, as well as volaceous
profiling approaches to plant disease detection, in order to design ground-based
sensors for monitoring plant health and disease in the field. 

\item A method for diagnosing image processing sickness was established after a thorough
analysis of the authors' findings, which included double-stranded ribonucleic acid
investigations, nucleic acid testing, and microscopy, among other plant diagnostic
instruments.

\item At the moment, computer vision is being utilised to identify plant diseases in a
variety of ways. One method for identifying illnesses is to employ colour extraction
services supplied by writers. In this study, the YcbCr, HSI, and CIELB colour models
were successfully employed to recognise noise from a variety of sources, including
camera flashes.

\item The method for generating characteristics could be utilised to detect plant diseases.
In recent tests, Patil and Bodhe used a threshold segmentation approach to estimate
the area of sugarcane leaves and its triangular threshold for lesion zones, with an
average accuracy of 98.60 percent.

\item The texture extraction feature can be used to diagnose plant issues. Patil and
Kumar created a method for detecting plant disease that uses textures such as inertia,
homogeneity, and correlation to construct the images gray-level cooccurrence pattern.

\item In maize leaves, they combined disease detection with colour extraction. All of these
factors come together to produce a potent set of traits for improving image and
classification quality. Some of the most frequent traditional character extraction
methods were discussed by the authors. Due to the rapid rise of artificial intelligence
research, the study focuses on the implementation of these approaches and strategies
(AI).

\item To recognise species of leaves, plagues, or illnesses, different neural network back-
feed propagation algorithms use one input, one output, and a hidden layer. The
authors recommended this model.

\item A pest and disease control software model for agricultural crops has been developed.

\item The researchers also use a new method for determining the location of damaged
cotton leaves that enhances system accuracy by 95 percent. The forward-looking neural
network and the particular swarm optimization (PSO) are used to demonstrate this.

\item Vector Machine Support can also detect and differentiate between various plant
ailments. This approach is used to diagnose sugar beet diseases, and it has a success
rate of 65 to 90 percent, depending on the kind and stage of the disease.

\item Other methods that combine feature extraction and the Network Ensemble can be
used to diagnose plant diseases (NNE). NNE generalises learning skills more broadly
through training and the integration of a number of neural networks. Because of its 91
percent accuracy, this approach was widely utilised to diagnose tea leaf problems.

\end{itemize}
In biology, bioinformatics, biomedicine, robotics, and 3D technologies, among other
fields of study, the authors have offered in-depth learning approaches to handle some
of the most difficult topics. This work applies deep learning techniques for plant
disease diagnostics by inventing and executing deep learning methodology. According
to a thorough analysis of the most recent literature, the researchers did not study a
comprehensive approach of evaluating leaf photos to diagnose plant ailments. The
next parts go through our deep CNN recognition technique.



